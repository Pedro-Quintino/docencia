\documentclass[a4paper,12pt,twoside]{report}
\usepackage[left=2cm,right=2cm,top=2cm,bottom=3cm]{geometry}
\usepackage{amsmath,amsthm}
\usepackage{graphicx,amssymb}
\usepackage[all]{xy}
\usepackage{amsfonts}
\usepackage{color}
\usepackage[brazil]{babel}
\usepackage[utf8]{inputenc}
\usepackage[colorlinks,linkcolor=black,hyperindex=black]{hyperref}
\usepackage{calligra}%Fonte
\usepackage{import}%Importar arquivos
\usepackage{calrsfs}%Letras caligráficas 
\usepackage{multicol}%Criar multiplas colunas
\usepackage{tikz}%Desenhar
\usepackage{tikz-cd}% diagrama de setas
\usetikzlibrary{positioning}%Desenhar 2 coordenadas plano cartesiano
\usepackage{verbatim}%Comentários de múltiplas linhas
\usepackage{imakeidx}% índice remissivo
\makeindex[columns=2]
%índice remissivo
\usepackage{hyperref}%referencias personalizada que vou utilizar para a lista de símbolos.
\usepackage{color,soul}%marca texto
\usepackage{multirow}%multiplas linhas pra tabela
%define a fonte padrão
\usepackage{helvet}
\renewcommand{\familydefault}{\sfdefault}
%pacotes para tabelas ajuste de tabelas
\usepackage{tabularx}
\usepackage{xltabular}
\usepackage{booktabs}
\usepackage{tcolorbox}


\pagestyle{headings}
\renewcommand{\baselinestretch}{1.5}
\newcommand{\C}{\mathbb{C}}
\newcommand{\A}{\mathbb{A}}
\newcommand{\R}{\mathbb{R}}
\newcommand{\N}{\mathbb{N}}
\newcommand{\Q}{\mathbb{Q}}
\newcommand{\Z}{\mathbb{Z}}
\newcommand{\K}{\mathbb{K}}
\newcommand{\M}{\cal{M}}
\newcommand{\U}{\cal{U}}
\newcommand{\F}{\cal{F}}
\newcommand{\As}{\mathcal{A}_s}
\newcommand{\Ca}{\mathcal{C}}
\newcommand{\G}{\mathcal{G}}
\newcommand{\J}{\mathcal{J}}
\newcommand{\T}{\mathbb{T}}
\newcommand{\FX}{F\langle X \rangle}

\newtheorem{obs}{Observação}[section]
\newtheorem{prop}{Proposição}
\newtheorem{defi}[obs]{Definição}
\newtheorem{teo}{Teorema}
\newtheorem{exe}{Exemplo}
\newtheorem{con}{Conjectura}
\newtheorem{lem}[obs]{Lema}
\newtheorem{cor}[obs]{Corolário}

\newenvironment{dem}[1][Demonstração]{\noindent\textbf{#1:} }{\hfill \rule{0.5em}{0.5em}}


\tcbuselibrary{theorems}

\newtcbtheorem{mytheo}{Teorema}%
{colback=green!5,colframe=green!35!black,fonttitle=\bfseries}{th}

\newtcbtheorem{mydef}{Definição}%
{colback=green!5,colframe=green!35!black,fonttitle=\bfseries}{th}


\newcommand{\tabitem}{\quad ~~\llap{\textbullet}~~}

\begin{document}
\noindent
\begin{tabularx}{\textwidth}{|X|X|}
\hline
\multicolumn{2}{|l|}{\textbf{Professor:} Pedro Quintino da Silva Neto}\\
\hline 
\textbf{Ano:} & \multirow{2}{*}{\textbf{Nível de ensino}:  }\\
\cline{1-1} \textbf{Turma:}  & \\
\hline
\multirow{1}{*}{\textbf{Carga Horária: }} & \textbf{Disciplina:} \\
\hline
\end{tabularx}

\begin{center}
    \textbf{Plano de Aula}
\end{center}
\vspace{-0.5cm}

\begin{xltabular}{\textwidth}{|X|}
\hline
\textbf{Tema: } \\
\textbf{Tópicos:}\\
\tabitem\\
\tabitem\\
\tabitem\\
\\\hline
\end{xltabular}

\begin{xltabular}{\textwidth}{|X|}
\hline \multicolumn{1}{|c|}{\textbf{Objetivos}} \\ \hline 
\endfirsthead
\hline \multicolumn{1}{|c|}{\textbf{Objetivos}} \\  \hline
\endhead
\hline \multicolumn{1}{|r|}{{Continued on next page}} \\ \hline
\endfoot
\hline
\endlastfoot
\textbf{Geral:}\\
\tabitem\\
\textbf{Específicos: }\\
\tabitem\\
\tabitem\\
\tabitem\\
\hline
\end{xltabular}


\begin{xltabular}{\textwidth}{|X|}
\hline \multicolumn{1}{|c|}{\textbf{Procedimentos didáticos/metodológicos}} \\ \hline 
\endfirsthead
\hline \multicolumn{1}{|c|}{\textbf{Procedimentos didáticos/metodológicos}} \\ \hline 
\endhead
\hline \multicolumn{1}{|r|}{{Continua na próxima página}} \\ \hline
\endfoot
\hline
\endlastfoot
``O conteúdo será apresentado por meio de aula expositiva dialogada. Essa apresentação será dividida da seguinte forma:''\\
\tabitem\\
\tabitem\\
\tabitem\\
\hline
\end{xltabular}

\begin{xltabular}{\textwidth}{|X|}
\hline \multicolumn{1}{|c|}{\textbf{Recursos materiais}} \\ \hline 
\endfirsthead
\hline \multicolumn{1}{|c|}{\textbf{Recursos materiais}} \\  \hline
\endhead
\hline \multicolumn{1}{|r|}{{Continua na próxima página}} \\ \hline
\endfoot
\hline
\endlastfoot
\tabitem\\
\tabitem\\
\tabitem\\
\hline
\end{xltabular}

\begin{xltabular}{\textwidth}{|X|}
\hline \multicolumn{1}{|c|}{\textbf{Avaliação}} \\ \hline 
\endfirsthead
\hline \multicolumn{1}{|c|}{\textbf{Avaliação}} \\  \hline
\endhead
\hline \multicolumn{1}{|r|}{{Continua na próxima página}} \\ \hline
\endfoot
\hline
\endlastfoot
\\
\end{xltabular}

\begin{xltabular}{\textwidth}{|X|}
\hline \multicolumn{1}{|c|}{\textbf{Referências Bibliográficas}} \\ \hline 
\endfirsthead
\hline \multicolumn{1}{|c|}{\textbf{Referências Bibliográficas}} \\  \hline
\endhead
\hline \multicolumn{1}{|r|}{{Continua na próxima página}} \\ \hline
\endfoot
\hline
\endlastfoot
\\
\end{xltabular}
\newpage

\begin{center}
\begin{underline}{\textbf{DESENVOLVIMENTO DA AULA}} 
\end{underline}
\end{center}

%%%%%%%%%%%%%%%%%%%%%%%%%%%%%%%%%%%%%%%%%%%%%%%%%%%%
\end{document}